\newpage
\section{现有方案}
\label{sec:prior_art}

零售管理的需求是跨越客户关系管理(CRM)、企业资源计划(ERP)、销售时点情报系统(POS)和“进销存”(进货、入库、销售)等多种运营管理领域\cite{tao2006erp-crm}的综合性的管理需求。同样,目前市面上较为常见的各类零售管理方案因此也覆盖了其中部分或全部的领域。

\subsection{“微盟”}

其中十分具有代表性的“微盟”是以微信平台为依托的,覆盖实体店面收银、库存管理、会员管理等各个功能的一个功能较为广泛的软件即服务(SaaS,Software as a Service)管理产品。在门店、导购数字化方案之外,“微盟”还提供私域电商\cite{qi2020private_ecommence}(由一个或多个特定实体主导的私有电子商务平台,有别于“淘宝”等较多不同实体进驻的公共平台)搭建和管理服务,有助于其客户利用平台整合度高的优势推进其自身的数字化和智慧化。

该方案是现有流行的方案中功能最完整、细分行业覆盖面最广的方案之一。然而,除了没有提供AI驱动的相关功能以外,微盟”和其他众多采用基于Web(HTML5)的用户界面的方案一样在易用性和可访问性上受到了限制,无法向用户提供更加便捷的管理体验;而与云服务的深度绑定则增加了用户使用的风险。

\subsection{“云POS”、“BOS Cloud”等}

另一个具有一定代表性的是伯俊科技公司的“云POS”和“BOS Cloud”等零售门店管理服务。该方案以方式种类丰富的POS为核心,提供传统桌面式收银台、手持式收银终端、自助收银终端等专用硬件。与“微盟”一样,该方案同样提供了一定程度的在线销售功能。不同的是“云POS”采用接入一般外卖、直播电商平台的方法实现线下门店的“线上化”,以此促进其客户的数字转型。值得注意的是,该平台具有接入其他第三方管理平台(如“微盟”)的方法,其供应商绑定风险较低。

\subsection{其他方案}

除了上述的两个方案,还有如“T+Cloud”实体店新零售管理系统、“茗匠”有人或无人零售收银管理系统、“金蝶”小微企业云服务平台和“致心零售”SaaS零售软件等整合了多个管理维度,提供零售行业一站式管理的各种不同方案。

\subsection{问题总结}

这些方案都在可以从某个或某些方面上增强对应零售业务的数字化、智能化程度,提高运营便利、自动化程度,进而改善产业生态和客户体验,但是它们都具备相似的问题:注重于将线下门店“线上化”、将电子商务“实体化”,把电子商务的优势整合到传统实体零售行业中,而一定程度上忽视了对这些技术和手段(无论线下还是线上)本身的进一步优化和发展(尤其是对人工智能技术的使用),进而在对经营者门店数字化工作便利性的增强和对消费者核心购物体验改善的工作上有所不足\footnote{上海微盟企业发展有限公司在2024年前后发布了“WAI”人工智能工具集合。根据其公布的少量资料,这些工具以面向经营者为主,尚未覆盖到零售业务的各方面。此外,这方面可供参考的资料稀缺,故“WAI”产品未被在正文提及。}。

此外,这些平台比较复杂的用户界面对使用者(零售行业从业者)的操作这些平台的技术能力提出了较高的要求,在产品易用性、可访问性方面存在一定的改进空间;前文列出的方案(还有许多其他较为常用的方案)通常依赖一个特定的云服务、一个或多个(服务提供商的)合作伙伴提供的云服务等,一定程度上造成了客户对该服务商的捆绑。

\subsection{本设计的改进}

本设计从服务于广大从业者、广大消费者的立场出发,在面向从业者的零售管理系统后台中添加了便于编写商品描述等资料的AI文案编写功能、在面向消费者的手机应用程序中添加了便于检索所需商品的AI驱动模糊搜索和AI导购助手等方便特性,还添加了门店结算终端可以使用的AI商品识别功能,从多个不同角度出发让人工智能的优势在传统零售行业充分绽放,与前文提及的系统相比,进一步地提升了从业者和用户的使用体验,降低了他们的使用负担。

此外,本设计提出的系统还通过充分利用不同平台的专门特性充分优化了核心组件、用户界面,相对于前文提及的其他方案,从业者可以更加简单地对系统进行控制,消费者可以更加快速地检索所需要的商品信息;本设计对应系统为从业者自行部署和二级服务商展开部署服务的情况做出考虑,一定程度上降低了服务和客户的捆绑,降低了客户利用该系统的长远风险。

通过利用人工智能这一“算法”,本设计所构建的AI驱动的零售商品全流程管理系统实现了以传统商业应用程序及其算法所无法实现的功能和特性,进而探究了从经营者、消费者多角度多层次提高零售行业运作水准的可能性,给出了将人工智能技术和传统零售行业相结合的切实方案、可行方案,为零售行业的数字化、智能化未来添砖加瓦。