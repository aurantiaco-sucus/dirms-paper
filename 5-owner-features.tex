\newpage
\section{商家端AI功能}
\label{sec:owner_features}

面向零售行业从业者的AI特性主要围绕着商品设计和图表分析展开。这两个业务板块实际上是以宣传为目的、以市场需要为导向的文本编纂;对营业中的各项数值在不同时间段中的变化趋势、不同周期的重复规律的研究。显然,不管是专业文本编纂还是销售数据挖掘都一定程度上超出了一般零售行业在店从业人员的能力范围。

对于连锁店或其他类型的多店面实体零售企业,门店可以通过系统联网、数据同步的方式将这两项工作转移到企业本体的研发、市场部门,运用专业文学工作者、数据分析人员的专业能力来缓解这样的矛盾,一定程度上也有助于同品牌店面行为的标准化和品牌形象的塑造。然而,这样的做法为每个特定的店面制定相应的经营策略的成本是较高的,并且将会对企业的相关资源造成较大的压力,但若是将每个店面的数据整合处理协同考虑,又会使得各个店面效率的上限下降。并且,这样的运营模式一定程度上忽视了在店管理人员、其他工作人员对店面及其周边市场情况的了解,没有充分利用到不同类型、不同位置工作人员的具体能力素养。

对于个体户

\subsection{商品设计辅助}

\subsubsection{编写文案}

\subsubsection{建议定价}

\subsubsection{提供额外参考}

\subsection{销售数据分析}

\subsubsection{图表生成}

\subsubsection{图表理解}

\subsubsection{营业策略建议}

\subsection{库存审计}