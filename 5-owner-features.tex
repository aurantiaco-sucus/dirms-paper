\newpage
\section{商家端AI功能}
\label{sec:owner_features}

面向零售行业从业者的AI特性主要围绕着商品设计和图表分析展开。这两个业务板块实际上是以宣传为目的、以市场需要为导向的文本编纂;对营业中的各项数值在不同时间段中的变化趋势、不同周期的重复规律的研究。显然,不管是专业文本编纂还是销售数据挖掘都一定程度上超出了一般零售行业在店从业人员的能力范围。

对于一般的连锁店或其他类型的多店面实体零售企业,门店可以通过系统联网、数据同步的方式将这两项工作转移到企业本体的研发、市场部门,运用专业文学工作者、数据分析人员的专业能力来缓解这样的矛盾,一定程度上也有助于同品牌店面行为的标准化和品牌形象的塑造。

然而,这样的做法为每个特定的店面制定相应的经营策略的成本是较高的,并且将会对企业的相关资源造成较大的压力,但若是将每个店面的数据整合处理协同考虑,再分发统一的策略,又会使得各个店面效率的上限下降。并且,这样的运营模式一定程度上忽视了在店管理人员、其他工作人员对店面及其周边市场情况的了解,没有充分利用到不同类型、不同位置工作人员的具体能力素养。

对于自由程度较高的连锁店、加盟店或个体户,实际商品文案与广告的编写、商品备货策略的制定、市场趋势的预测和最终商业决策的施行一般均由该店面所对应管理者(如店长、店主等)进行。明显地,这对相关人员的文学素养、计算机等数字设备使用的熟练程度和“洞察力”(在特定情况下推理、判断并反映事物之间的因果关系的能力)提出了较高的要求。

\subsection{商品设计辅助}

\begin{figure}[htbp]
	\centering
	\includegraphics[width=0.8\textwidth]{./imgs/ai-writeup.png}
	\caption{AI商品设计辅助的执行流程图:其中上部两个的虚线框中内容所示不同输入和执行过程理论上可相互替换,本设计采用左侧部分的流程。}
	\label{fig:ai-writeup}
\end{figure}

为了缓解从业者在执行传统零售数字化的过程中遇到的难以编写高质量商品文案的问题,该设计包含一项特色AI功能:商品设计辅助。从业者提供商品对应的图片,也可以继续提供额外的参考资料,而后AI便可以根据这些信息产生对应的商品资料,从业者既可以根据实际情况直接采用生成结果,也可以在结果的基础上进行人工修改,较为具体的执行过程如图 \ref{fig:ai-writeup} 所示。

\subsubsection{编写文案}

与本设计的执行流程,利用AI编写出高质量的文案的条件可以归纳为以下几点:

\begin{itemize}
    \item LLM类型适合商业宣传性文案的编写
    \item LLM上下文窗口(context window)足够大
    \item LLM文字理解、逻辑推理、文学性能足够优越
    \item 使用的LLM具备多模态(multi-modal)特性或所使用的多个LLM中包含具有该特性的模型
    \item LLM具备足够高的节制度(moderation)
    \item 具备多模态特性的LLM图像细节(包括文本)的识别能力足够好
    \item 提示词(prompt)诱导LLM生成正确的、可自动处理的回复
    \item 提示词诱导LLM从图像提取更多信息
    \item 提示词最大程度上限制LLM在结果中包含臆想(hallucination)内容
\end{itemize}

对于文本生成的部分,笼统地说,任何适用于一般用途(也就是不是为某特定专业用途设计)的LLM都一定程度上具备编写商品文案的条件,只要合成合理的提示词或用于补全的文本就可以使用。本设计所对应的应用程序实现使用了阿里巴巴公司于“百炼”生成式AI服务平台提供的\verb|qwen-plus|商业大语言模型。若选用的LLM为进行对话(Instruct)进行过调优(如\verb|qwen-plus|),那么比较合理的输入是与“文案编写助手”的对话。本设计程序中生成商品描述提示词对话如下:

\begin{itemize}
    \item[] \textbf{系统:}你是一个处理商业资料的智能助手,不会输出问题答案以外的任何信息。
    \item[] \textbf{用户:}请根据以下提示编写可以吸引到潜在客户的网店产品介绍文案,400字以上,并且整理成适当长度的自然段。但是请尽可能不要对图片没有提及的商品情况和商店服务作出假设。以下是该产品图片的描述:\textit{(产品图片描述)}\dots
    \begin{itemize}
        \item[] \textbf{(提取图片文字时)}\dots 以下是从这个图片中发现的文字(若图片中没有文字,则可能会是无意义信息):\textit{(产品图片文字)}
        \item[] \textbf{(提供额外上下文时)}\dots 以下是店主附加的信息,请参考:\textit{(附加参考材料)}
    \end{itemize}
\end{itemize}

从图 \ref{fig:ai-writeup} 可见用于编写商品描述的材料可以来源于用户直接输入对目标商品的描述性文字,和交由额外模型处理的图片输入具有相似的效果。但是,采用这种输入方式的文案编写过程对营业者提出了新的要求,某种程度上相当于将对文学素养、宣传材料撰写能力的要求化为了对产品特征表达能力的要求,虽然同样是对“高门槛”问题的缓解,但不如输入图片的处理方式彻底\footnote{实际上,输入图片同样会对用户提出图像质量把控的要求。但明显地,一般用户满足该要求的概率相对而言是更高的,并且学会拍摄清晰照片的难度也更低。}。

如果使用性能较为强大的多模态模型(如开源的“千问”系列 \verb|Qwen2.5-VL| 模型),实际上可以将图片理解、描述的步骤和实际商品详情生成的步骤整合。在本设计的实现中考虑到大规模的多模态模型云服务对应的成本较高,并且输出速度一般慢于仅文本的大模型,故将两个步骤拆分,运用不同的模型处理。该设计中运用的视觉理解多模态大模型为“百炼”的商业大模型 \verb|qwen-vl-plus| ,程序中生成商品图片描述的提示词对话如下:

\begin{itemize}
    \item[] \textbf{系统:}你是一个处理商业资料的智能助手,不会输出问题答案以外的任何信息。
    \item[] \textbf{用户:}
    \begin{itemize}
        \item[] \textbf{(商品图片数据)}
        \item[] \textbf{(文字信息)} 该图片中包含一个商品,请用尽可能详细的语言描述该商品的外观特点,以便于后续无法查看该图片的其他模型处理。
    \end{itemize}
\end{itemize}

此外,程序中从商品图片中提取文字所使用的模型为“百炼”提供的商用特化模型 \verb|qwen-vl-ocr| ,提示词对话如下:

\begin{itemize}
    \item[] \textbf{系统:}你是一个处理商业资料的智能助手,不会输出问题答案以外的任何信息。
    \item[] \textbf{用户:}
    \begin{itemize}
        \item[] \textbf{(商品图片数据)}
        \item[] \textbf{(文字信息)} 该图片中包含一个商品,请发现该商品外观中包含的所有文字。但如果没有发现文字,也请如实回答。
    \end{itemize}
\end{itemize}

\subsubsection{建议定价}

对于一些特殊的零售细分行业、比较不常见的店面地段和无标准或习惯定价的商品种类,商品的最优定价并不明显。针对这一类较为特殊的情况,该设计提供一个参考定价生成功能,可以利用前文提及的商品描述和(可选的)附加参考材料来生成一个用于参考的商品价格。鉴于定价的过程逻辑性比较强,虽然该过程理论上可以采用一般大模型作为生成价格的算法,此处作为特殊情况采用了具备内省\footnote{即所谓“深度思考”(deep thinking)功能}(reflection)能力的大语言模型。这类模型中十分知名的是深度求索公司发布的开源模型“DeepSeek R1”,但为了缓解这类模型输出结果较为缓慢(模型较大并且具有思考过程)的问题,该设计采用(云上托管的)阿里巴巴发行的开源大模型“QwQ”。

显然,该过程需要输出的是价格的表示(于第 \ref{sec:foundation} 部分中提及的RmCore模块中的Price类型),但由于众所周知大语言模型的输出格式是较为随机的,此处采用许多常见文本模型具备的将输出规范为特定结构的JSON对象的功能。经过格式优化适合浏览的提示词对话如下:

\begin{itemize}
    \item[] \textbf{系统:}你是一个处理商业资料的智能助手,不会输出问题答案以外的任何信息。
    \item[] \textbf{用户:}请根据以下信息猜测该商品的价格。输出格式:
    \item[] \verb|{|
    \begin{itemize}
        \item[] \verb|"unit": 单位个或单位克|
        \item[] \verb|"price": 单位价格人民币分|
        \item[] \verb|"pricing": 计价方式|
    \end{itemize}
    \item[] \verb|}|
    \item[] 例如“可乐一罐卖3元”表示为
    \item[] \verb|{"unit": 1, "price": 300, "pricing": "Package"}|,
    \item[] “苹果一斤卖6元”表示为
    \item[] \verb|{"unit": 500, "price": 600, "pricing": "Weight"}|,
    \item[] 还请不要输出任何无关内容。以下是该产品图片的描述:\textit{(产品图片描述)}\dots
    \begin{itemize}
        \item[] \textbf{(提取图片文字时)}\dots 以下是从这个图片中发现的文字(若图片中没有文字,则可能会是无意义信息):\textit{(产品图片文字)}
        \item[] \textbf{(提供额外上下文时)}\dots 以下是店主附加的信息,请参考:\textit{(附加参考材料)}
    \end{itemize}
\end{itemize}

由于JSON格式的技术限制,\verb|pricing|枚举类型部分无法限制大模型的输出为合法值。桌面商家端应用程序此部分在后期反序列化进一步进行格式检查。

\subsubsection{提供额外参考}
前文提及用户可以在生成商品详情、标题和(可选地)价格时可以选择提供额外的资料给AI

\subsection{销售数据分析}

\subsubsection{图表生成}

\subsubsection{图表理解}

\subsubsection{营业策略建议}

\subsection{库存审计}