\makecover

\maketitleinfo

\setcounter{page}{2}

\begin{chineseabstract}
	当今时代人工智能技术、人工智能模型训练和推理行业发展迅速,各行各业都在积极利用人工智能技术助力各个产业的数字化、智能化进步,进一步推动社会生产力的创新发展。在这股AI(人工智能)浪潮之中,本就因为电子商务的兴起而态势疲弱的传统零售行业面临进一步落后的风险。但是具有实际体验并与商品进行各种互动等独有优势的传统零售行业仍有进一步发展的空间、数字化转型的必要。本设计以服务于广大消费者、零售从业者为宗旨,以构建实际零售管理系统为实验方法,在商品文案设计、商品检索、商品推荐与咨询、商品结算、库存管理、销售数据分析等方面,探究并提出了一系列将时代前沿AI技术与传统零售行业管理和服务系统相结合的具体办法。
\end{chineseabstract}



\begin{chinesekeywords}
	人工智能;零售;大语言模型;实体经济
\end{chinesekeywords}



\begin{englishabstract}
	Nowadays the advancement of AI (artificial intelligence) technologies and AI model training and inference industry are significant. Most industries and areas are actively leveraging AI technologies to empower the digitalization efforts, boosting the creative step forward of our society. However, despite the emerging AI innovations, traditional retail businesses, being already behind the waves because of the rising e-commerce industry, face the risk of further decline. But it is also apparent that having the unique feature of allowing end-users to experience and interact with products demonstrates that retail businesses could and should be further advanced and digitalized. This project design, with the consumers and retail owners in mind, in the way of building an functional retail management system, researches and purposes a set of concrete methods to fuse AI and traditional retail together, including the design of product write-ups, product searching, product recommendation and consulting, check-outs, inventory management and analysis of sales data.
\end{englishabstract}


\begin{englishkeywords}
	Artificial Intelligence, Retail, Large Language Model
\end{englishkeywords}


\newpage