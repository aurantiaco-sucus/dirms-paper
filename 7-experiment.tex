\newpage
\section{实验}
\label{sec:experiment}

\subsection{数据准备}
为了更好地开展商品信息相关的实验、测试,现以Product-10K数据集\cite{bai2020products10k}中的商品图片为基础,利用本地部署的多模态LLM批量生成这些商品对应的商品信息,再将这些信息输入到中心服务器中,以达到替代实际商品商品信息的目的。该数据集中的图片以存货单位(SKU)、商品类别为分类方式。在本实验中属于同一SKU的图片将被视为同一个商品的图片,每个SKU都会被随机挑出一个图片作为其代表交由LLM进行处理。文字部分利用的LLM是30亿参数版本 \verb|qwen2.5| 。

本实验利用本地部署的80亿参数版本 \verb|minicpm-v| 大模型进行商品图像描述的编写,使用的提示词对话如下:

\begin{itemize}
    \item[] \textbf{用户:}
    \begin{itemize}
        \item[] \textbf{(商品图片数据)}
        \item[] \textbf{(文字消息)} 细致地描述这个图片中的内容,以便其他模型利用这段描述进行进一步推理。
    \end{itemize}
\end{itemize}

然而,对生成结果(及其对应图片)的人工抽查显示数据集中某些图片更加接近生活照片,难以被断定为商品图片。因此,以下的提示词对话被用于检测图片是否代表一个商品:

\begin{itemize}
    \item[] \textbf{用户:} 以下是从一个图片产生的说明,请根据这段说明判断该图片是否表示一个商品(布尔值字段verdict):\textit{(商品图片描述)}
\end{itemize}

以上提示词在使用的时候加入了对模型输出的限制(具有布尔值字段 \verb|verdict| 的JSON对象)。经过筛选之后的商品图片描述其后进入商品描述生成的过程:

\begin{itemize}
    \item[] \textbf{用户:} 请编写商品描述,商品描述应该尽可能有吸引力,并且不要输出商品描述之外的内容。以下是商品对应图片的描述:\textit{(商品图片描述)}
\end{itemize}

最后,生成的商品描述被用于进一步生成商品标题:

\begin{itemize}
    \item[] \textbf{用户:} 为这段商品介绍产生一个言简意赅的商品标题:\textit{(商品描述)}
\end{itemize}

生成的商品标题、描述等数据其后被收集起来,并且与其图片相绑定。之后,这些数据被通过第 \ref{sec:foundation} 部分提及的服务器API传输到数据库中,用于后续试验。

\subsection{商家端}

\subsubsection{桌面应用}

\subsubsection{手机应用}

\subsubsection{数据集管理程序}

\subsection{顾客端}

\subsubsection{手机应用}

\subsubsection{结算程序}