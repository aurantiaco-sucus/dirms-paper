\newpage
\section{顾客端AI功能}
\label{sec:guest_features}

面向最终消费者的AI功能主要围绕着发现心仪商品和结算所购买商品的场景展开。顾客发现商品的方式大致可以分为目的导向的和非目的导向的商品发现方式,其中非目的导向的商品发现方式与商户的宣传手段及其有效性关联较大,很大程度上取决于顾客是否浏览到该商铺对应宣传内容(实体、在线广告、客户群等)和对于宣传内容的实际吸引力。宣传内容的曝光可以由从业人员对社交媒体的参与来实现,而宣传内容可以借助上一部分提及的AI商品文案起草特性来辅助。

目的导向的商品发现方式主要包括用户对商品进行搜索的过程。对于叫法比较单一、名称好记没有歧义的商品,简单分词-匹配的关键词搜索功能是可以满足需要的。然而,时间情况下商品名称匹配的问题可能远复杂于理想的情况。例如笼统和详细说法的区别:“可乐”和“苏打水”都可以叫作“汽水”,但这三个词语之间却无法直接相互匹配,并且若是为此将“汽水”拆为“汽”和“水”,不但仍然无法和“可乐”匹配,还可能会误匹配到与如“水”、“水汽”等词语相关的其他商品。

为了更好。。。搜索引擎

结算。。。商品识别

\subsection{导购助手}

\subsubsection{对话型生成式AI}

\subsubsection{输出风格}

\subsubsection{搜索关键词提取}

\subsubsection{商品搜索}

\subsection{称重商品识别}

\subsubsection{数据准备}

\subsubsection{图像处理}

\subsubsection{AI图像分类}

\subsection{模糊搜索}

\subsubsection{商品数据处理}

\subsubsection{AI近义词搜寻}

\subsubsection{词典构建}

\subsubsection{搜索算法}