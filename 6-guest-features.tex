\newpage
\section{顾客端AI功能}
\label{sec:guest_features}

面向最终消费者的AI功能主要围绕着发现心仪商品和结算所购买商品的场景展开。顾客发现商品的方式大致可以分为目的导向的和非目的导向的商品发现方式,其中非目的导向的商品发现方式与商户的宣传手段及其有效性关联较大,很大程度上取决于顾客是否浏览到该商铺对应宣传内容(实体、在线广告、客户群等)和对于宣传内容的实际吸引力。宣传内容的曝光可以由从业人员对社交媒体的参与来实现,而宣传内容可以借助上一部分提及的AI商品文案起草特性来辅助。

目的导向的商品发现方式主要包括用户对商品进行搜索的过程。对于叫法比较单一、名称好记没有歧义的商品,简单分词---匹配的关键词搜索功能是可以满足需要的。然而,时间情况下商品名称匹配的问题可能远复杂于理想的情况。例如笼统和详细说法的区别:“可乐”和“苏打水”都可以叫作“汽水”,但这三个词语之间却无法直接相互匹配,并且若是为此将“汽水”拆为“汽”和“水”,不但仍然无法和“可乐”匹配,还可能会误匹配到与如“水”、“水汽”等词语相关的其他商品。

为了在一定程度上解决该问题,本设计包含模糊搜索引擎项目“探寻”及其对应词典处理脚本。该子项目采用“结巴”分词库\cite{sun_fxsjyjieba_2025,messense_messensejieba-rs_2025}进行分词,并且利用大语言模型对每个词语的近义词进行枚举,最后将各个来源的处理结果整理为高查询效率的格式在为中文优化的自定义搜索算法中进行部署,以此在消耗比较少的计算资源的情况下达到较高的搜索速度和(中文)搜索的准确率,有助于最终消费者更好地进行目的导向的商品发现活动,推动消费体验、营业质量提升。

然而性能更加强大的模糊搜索系统无法解决在许多情况下顾客不知悉需要搜索的关键词(及其近义词)的问题,这种情况下,顾客可能甚至并不清楚自己实际需要的商品。这个问题较为明显的解决思想是使得“商品发现”相关功能具有理解消费者对其需求的描述的语义并将需求内容对应于特定商品信息,或者为此生成对应的搜索语句提供给用户进行检索(或自动运行检索)。

为了解决这种情况带来的问题,该设计的顾客端移动应用程序包含AI导购助手模块。该模块利用经过特定提示词引导的多轮LLM对话及单次LLM调用,分别营造与顾客进行导购交流、导购向用户提出购买建议,如此往复的体验;从导购对用户的回复中提取出适用于检索的关键词语句,以此实现消费者只要合理形容需求,便可检索到对应商品的功能。

AI识别计重商品结算模块是该设计对一般传统实体零售流程的另一个改进。通过利用AI物体识别算法,在零售管理系统中整合物体识别AI模型的训练数据采集、标注等操作对应的用户界面,自动化模型训练和部署的过程;在结算终端中整合AI物体识别前端软件及摄像头、质量传感器等硬件来实现营业者轻松部署AI物体识别模型,最终用户轻松自助结算计重商品,去除计重商品结算过程对店员参与的要求。

\subsection{导购助手}

\subsubsection{对话型生成式AI}

\subsubsection{输出风格}

\subsubsection{搜索关键词提取}

\subsubsection{商品搜索}

\subsection{称重商品识别}

\subsubsection{数据准备}

\subsubsection{图像处理}

\subsubsection{AI图像分类}

\subsection{模糊搜索}

\subsubsection{商品数据处理}

\subsubsection{AI近义词搜寻}

\subsubsection{词典构建}

\subsubsection{搜索算法}