
\section{结论}

本设计项目立足于传统实体零售行业,致力于探究在传统零售行业中应用、融入乃至于整合人工智能技术,利用人工智能的便利增强经营者的营业能力并降低运营门槛,将更便利的、更有亲和力的实体购物体验带给消费者。

在实现一套完整的基本零售管理系统的基础上,本设计创新地加入了AI商品文案编写、AI驱动的模糊搜索、AI导购助手和AI商品识别等功能,提高了从业者的营业效率和增强了顾客的购物体验。将大模型与基本系统深度整合,本设计探索了“AI+零售”的未来可能性,一定程度上较现有方案更具备创新性。

在模型生成文字质量的把控和整体AI智能程度、对零售业务了解程度上,本设计尚有可以改进之处,未来通过改善提示词、换用效果更好的模型,或者通过采用智能体架构(Agentic AI)等先进推理框架和手段可以期望这些问题的解决。

\subsection{AI相关的新功能}

本设计为传统的零售管理系统添加了许多AI相关的功能,功能的类别和设计思想、具体实现和试验效果在前文中已经提及,但为更好总结设计、实现和测试的情况,在此从理解AI在该系统中担任的角色的角度方面再次对实现的功能作出概括。

AI模型可以是“万能函数”。本设计所实现的商品信息生成功能,实质上是输入商品图片等收集难度低的数据,输出商品描述和标题等编写(或收集)难度高的数据,以此来降低从业者数字化转型、维护的负担的功能。也就是从易于获取的数据中提取(由人或算法)较难得出的推论。模糊搜索算法中近义词词典的AI生成过程也是同样的道理,利用固定的词典将会面临较大的局限性,而利用互联网搜索效率较低,但利用从互联网信息训练而来的AI模型则较好地达到了效果和效率上的平衡。

利用AI不能(高效)解决全部问题,但合理利用AI模型有助于在传统算法、函数和人工对接难以处理的情况(常常和自然语言、音视频相关)下达到效率、效果的双重改进。这对于历史上对人工的重复性工作依赖极强的传统实体零售行业,未来的突破和转型具有较大的利用价值。

AI也可以是人工的强化。通过利用AI辅助商品的生成,原本可能在文案编写能力较弱的从业者也可以在不需要外界(如专业写手)的帮助下编写出辞藻丰富的宣传文本。通过利用AI完成导购任务,没有配备专门导购人员的较小商店也有机会达到大卖场的营运效果,这具有大幅强化商店“人-货-场”相匹配的程度的潜力。此外,利用AI识别商品还使得去除店员在称重工作上的参与成为了可能。

当前的AI一般还并不足以或并非被设计用于全面地替代人工,但通过人与AI的协同工作,常常受到限制的人的能力能够得到极大的拓展,同时人的思考也可以一定程度上规避、缓解AI时常会出现的错误。如此,零售行业从业者的负担有望被减轻,同时执行运营事务的门槛也可以得到进一步降低。

\subsection{基本系统的优化}

本设计除了为零售管理系统添加许多AI相关的新功能,还在基本的系统上作出了一些有利于提高营业效率、降低运营成本的改进。其中包括在服务器、数据系统的改善和在用户界面上的改善。

服务器上,本设计采用rqlite分布式数据库系统,充分考虑到了对数据冗余性和多服务端场景下数据库同步的需要。的在数据系统上,本设计采用了库存与进货批次相绑定的做法,保证每次进货都可以追踪到对应的进货批次和进货计划,降低检查、分析的难度。

用户界面上,本设计遵从充分利用不同设备类型固有特性、优势的思想,在桌面端用户界面充分利用鼠标左右键操作和键盘快捷键,提高用户界面的效率和可访问性;在移动端用户界面利用滑动和触摸操作,在许多场景下给用户更多的操作方式选择。

\subsection{“零售商品管理系统”}

本设计实现的是“AI驱动的零售商品全流程管理系统”,没有以已有的管理系统方案作为基底的原因包括现有方案开放程度不足、开源方案对零售行业特化不足(一般为宽泛的CRM、ERP系统)等问题。但最为重要的是,从实验可以看到本设计包含的基本系统虽然体量较小,功能不及商业方案完整,但具备自定义方案的最大优势:与项目目标(AI驱动)的高度协调。本设计所对应系统一定程度上做到了与AI的深度整合和绑定,是目前多数方案所没有做到的。

\subsection{改进空间}

该设计得益于自定义的基本零售管理系统,实现了与AI的高度协同。但也因为开发了自定义的基本系统花费较多的精力和资源,许多AI功能尚有进一步优化、发掘的空间,还有许多潜在的AI功能遗憾地没有得到实现。

较为显著的一个问题是AI生成文字的高度不确定性。AI所产生的输出具有一定的随机性,在提示词或提供的商品信息上的细微的变化也可以使得结果变得完全不一致。不同模型的不同推理、上下文理解性能也可以对生成结果产生极大的影响。虽然带有内省功能的模型(如DeepSeek-R1)可以一定程度上缓解这个问题,但这样会带来极其长的推理时间,许多情况下是不可接受的。这将会是AI与零售相结合必须攻克的一个难关。

另一个较为显著的改进点是AI对零售门店整体情况(门店类型、细分领域等)的总体把握程度。使用完全不了解营业情况的AI相当于让对门店一无所知的人参与管理,这样推理结果的质量必然是较低的。但笼统地直接将门店相关信息、统计数字直接拼接到一个字符串交到大模型的简单方法又会使得效率难以提高,严重浪费计算资源。

合理推断,通过利用带有内省、工具调用功能的AI和智能体(Agent),让AI学会根据需要获取业务资料,统计业务信息,这种问题可以得到一定的改善。

\printbibliography[heading=bibintoc]

\section*{致谢}
\addcontentsline{toc}{section}{致谢}

本设计到此便即将结束,我希望对在这个项目进行设计、开发实现、测试和论文编写过程中给予我帮助的所有人表示由衷的感谢。

首先,我必须感谢该设计的指导老师闫红洋副教授。她严谨、敬业、关心学生成长和未来的教授形象为我和许多其他同学树立了榜样,我相信她的专业精神将成为我们未来工作生活的重要参考和努力目标。

同时,我需要感谢在该过程中向我给予支持、提出建议的其他老师、同组同学、同班同学和其他朋友们,没有他们的热心帮助,我的工作生活将困难许多。我还要感谢我的家人在生活上的各种帮助,没有你们,也就没有今天的我。