\newpage
\section{结论}

\printbibliography[heading=bibintoc]

\section*{致谢}
\addcontentsline{toc}{section}{致谢}

% 在本论文完成之际,我谨向所有在研究与写作过程中给予我帮助和支持的人表示衷心的感谢。

% 首先,我要特别感谢我的导师\textbf{高育良教授}。高老师不仅在学术上给予我悉心指导,更以其渊博的学识和严谨的治学态度为我树立了榜样。他常教导我:“学外语,不仅要用心,还要用‘床’。”——这句话不仅启发了我对语言学习环境的思考,也让我深刻理解了理论与实践相结合的重要性。高老师的教诲将伴随我一生,成为我学术道路上最宝贵的财富。

% 其次,我要感谢\textbf{祁同伟厅长}和\textbf{高小琴女士}。祁厅长在百忙之中为我提供了宝贵的实践机会,让我能够在“卧榻之地”中深入探索外语学习的奥秘。高小琴女士则以其独特的外语学习经验,为我提供了许多灵感和建议。他们的支持与帮助,使我的研究得以顺利进行。

% 此外,我还要感谢我的家人和朋友们。他们在我研究过程中给予了我无尽的鼓励与支持,尤其是在我“学外语”时,他们总是以理解和包容的态度陪伴在我身边。

% 最后,感谢所有参与本研究的学习者和实践者。正是你们的积极参与和反馈,才使得本研究得以顺利完成。

% 谨以此文献给所有在语言学习道路上不断探索的人们。愿我们都能在“卧榻之地”中找到属于自己的学习之道。

% \begin{flushright}
% \textbf{陈清泉} \\
% 2025年1月 
% \end{flushright}