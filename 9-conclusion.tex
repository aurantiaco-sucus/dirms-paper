\newpage
\section{结论}

本设计项目立足于传统实体零售行业,致力于探究在传统零售行业中应用、融入乃至于整合人工智能技术,利用人工智能的便利增强经营者的营业能力并降低运营门槛,将更便利的、更有亲和力的实体购物体验带给消费者。

在实现一套完整的基本零售管理系统的基础上,本设计创新地加入了AI商品文案编写、AI驱动的模糊搜索、AI导购助手和AI商品识别等功能,提高了从业者的营业效率和增强了顾客的购物体验。将大模型与基本系统深度整合,本设计探索了“AI+零售”的未来可能性,一定程度上较现有方案更具备创新性。

在模型生成文字质量的把控和整体AI智能程度、对零售业务了解程度上,本设计尚有可以改进之处,未来通过改善提示词、换用效果更好的模型,或者通过采用智能体架构(Agentic AI)等先进推理框架和手段可以期望这些问题的解决。

\printbibliography[heading=bibintoc]

\section*{致谢}
\addcontentsline{toc}{section}{致谢}

本设计到此便即将结束,我希望对在这个项目进行设计、开发实现、测试和论文编写过程中给予我帮助的所有人表示由衷的感谢。

首先,我必须感谢该设计的指导老师闫红洋副教授。她严谨、敬业、关心学生成长和未来的教授形象为我和许多其他同学树立了榜样,我相信她的专业精神将成为我们未来工作生活的重要参考和努力目标。

同时,我需要感谢在该过程中向我给予支持、提出建议的其他老师、同组同学、同班同学和其他朋友们,没有他们的热心帮助,我的工作生活将困难许多。我还要感谢我的家人在生活上的各种帮助,没有你们,也就没有今天的我。