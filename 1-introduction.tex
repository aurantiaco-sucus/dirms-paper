\section{前\hspace{1em}言}

近几年来,人工智能(AI)技术有了前所未有的深入发展。从上世纪50、60年代“机器学习(Machine Learning)”\cite{samuel1959machine_learning}概念被提出时它初次登上历史舞台,到人们利用图像处理单元(GPU)等专用硬件进行处理\cite{raina2009deep_unsupervised_learning}如AlexNet\cite{krizhevsky2017alexnet}和ResNet\cite{he2016resnet}等规模成指数级别增长的深度学习(Deep Learning)模型,再到2017年Transformer模型\cite{vaswani2023attention_is_all_you_need}掀起自然语言处理领域变革,AI领域已然有了长足的进步。而近几年从OpenAI ChatGPT\cite{openai2022chatgpt}引爆大语言模型(LLM)热潮到以DeepSeek为代表的一众大幅降低部署、使用成本的开源大模型走进人们生活不难看出,AI技术势不可挡,并且在可以遇见的未来还会进一步发展壮大。

人工智能技术既是一个独立的领域,又是其他行业和领域进一步深入发展\cite{su2025ai_plus_concrete_economics}、进行数字化转型\cite{xie2025ai_plus_digitization}的不可或缺的一部分。例如多模态(multi-modal)的大语言模型将图像、音频等不同媒介的信息与一般大语言模型的文字信息连接起来,形成了“看得懂”、“听得懂”的大(语言)模型\cite{radford2022multimodal_speech}。从智能制造到智慧医疗,人工智能在垂直领域中逐步渗透,与不同行业、工业相结合,使其得到了新的发展力量。

与朝气蓬勃的人工智能产业现成鲜明对比的一个领域是传统实体零售行业。改革开放以来,直到电子商务(电商)产业兴起以前,实体店铺几乎是民众购买不同产品的唯一方式,担任了将商品从设计、生产和批量分发的企业转移到最终用户的桥梁的角色,既是产品供给的终点,也是收益反馈的起点。以淘宝为代表的网上购物平台(也就是面向最终消费者的电子商务平台)兴起之后,产业链成本、消费便利性等种种因素使得消费者愈发青睐网上购物,足不出户便能选购喜爱的产品。即便没有自主选购的意愿,吸引人的商品也会从各种不同的广告推荐渠道来到消费者眼前。如此突出优势,与近年波动的经济环境和不安定的地缘政治情况,化为了许多中小型实体零售企业、个体户的运营压力,甚至使得其中的许多面临不得不终止运营的极端情况。

然而,实体零售行业仍有无法被取代的优势。消费者在参与线下购物的过程中,可以通过与商品的近距离互动来产生对其直观的印象,这种“零距离”的、有着天然信任的购物体验无法复制;网上购物的方法割裂了顾客支付商品价格和接受实体商品的过程,对物流有着一定程度的依赖,而实体零售则可以购买当时直接获得对应实体产品。此外,线下购物的过程同时也可以是社交的过程,可以营造独特的社会价值,增强消费者的生活体验。

为了在充满激烈竞争的经济环境下保持甚至提升自身的地位,实体零售行业需要积极进行自身的数字化、智能化转型,通过各种不同方法充分发掘实体零售的独特优势。在这样的情势下,“新零售”\cite{zhao2017new_retail,du2017new_retail}、“智慧零售”\cite{liao2019intelligent_retail}等概念、预想应运而生,“人-货-场”匹配的最优化\cite{wang2018person_good_place}受到广泛研究探讨。但是,这样的转型目前一般只有市场头部企业开始实施,体量较小的企业和个体户尚无资本和技术能力展开;并且其中较为重要的一个方面,实体零售与人工智能技术的结合,还有待深入开发。

本设计项目立足于传统实体零售行业,尤其是成本上受到较大限制的中小型企业和个体户,对产业数字化、智能化的需求日渐急迫的当下,致力于探究在传统零售行业中应用、融入乃至于整合人工智能技术,利用人工智能的便利增强经营者的营业能力并降低运营门槛,将更便利的、更有亲和力的实体购物体验带给消费者。具体来说本项目在实现一套基本可用的分布式零售管理基础设施、管理软件的基础上,利用不同类型的人工智能技术实现了以下几个不同的功能模块:

\begin{enumerate}
    \item 服务器部分
    \begin{enumerate}
        \item 智能分词技术、近义词搜寻技术驱动的关键词搜索引擎
    \end{enumerate}
    \item 管理端部分
    \begin{enumerate}
        \item 大模型驱动的智能商品文案编写助手
        \item 大模型驱动的智能业务图表分析、运营建议模块
        \item 基于智能条码识别、扫描技术的点货功能
    \end{enumerate}
    \item 门店端部分
    \begin{enumerate}
        \item 基于图像分类的智能商品识别功能
    \end{enumerate}
    \item 客户端部分
    \begin{enumerate}
        \item 大模型驱动的多轮对话、搜索推荐智能导购助手
    \end{enumerate}
\end{enumerate}

该部分之后的文章内容从介绍和分析该领域(零售管理)的现有方案(章节 \ref{sec:prior_art})开始,其后从整体角度对该项目所实现系统的架构设计作出解释说明(章节 \ref{sec:architecture}),然后分别对各个模块的具体设计、实现方案进行详细的描述,再之后对该系统相关的测试和运行效果进行列举和说明。最后对该系统的实现效果、未来改进空间等话题进行讨论,进而结合这个项目的情况对该领域的未来作出预测来对该文章收尾。

% \begin{figure}[htbp]
% 	\centering
% 	\includegraphics[width=0.8\textwidth]{./imgs/外语学习证明.jpg} % 图片路径
% 	\caption{作者本人“外语学习”实践场景:卧榻之地与语言习得的跨界融合\ 图源\cite{A橘色的海2025(补)假如陈清泉真的在学外语}} % 图片标题
% 	\label{fig:learn_english} % 图片标签
% \end{figure}