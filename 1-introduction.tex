\section{前\hspace{1em}言}

人工智能在2018年

网络、物流如此发达。越来越多的用户选择使用在线购物平台,足不出户地选购自己喜爱的产品。也因此,与在线购物平台这样的“新型服务业”相比,实体零售行业就显得十分苍白无力。但是如实际观察产品外观、触碰产品等操作,挑拣熟食、水果、蔬菜和肉类等做法在在线购物平台上难以开展,实体零售仍然有不可动摇的重要地位。因此,使传统零售业进行数字化转型,通过利用人工智能(AI)等最新技术,实现“人、货、场”的最优化匹配,向“智慧零售”、“新零售”转变就显得尤为重要。实际上现在改革已经迈入进行时,不少零售场所和其服务方式都变得“生活化”,在选购产品之外给消费者提供了额外的价值,不少零售商转变其盈利模式,独立的经营方式、品牌和渠道手段被广泛重视。

% \begin{figure}[htbp]
% 	\centering
% 	\includegraphics[width=0.8\textwidth]{./imgs/外语学习证明.jpg} % 图片路径
% 	\caption{作者本人“外语学习”实践场景:卧榻之地与语言习得的跨界融合\ 图源\cite{A橘色的海2025(补)假如陈清泉真的在学外语}} % 图片标题
% 	\label{fig:learn_english} % 图片标签
% \end{figure}