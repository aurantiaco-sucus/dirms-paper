\newpage
\section{技术概括}
\label{sec:tek-summary}

本设计开发和采用了多种类型的人工智能、商业、网络及软件工程相关技术和设计思想,其中较为突出的技术要点如下。

\subsection{推理抽象化、模型的量化和封装}

为了将人工智能(尤其是大语言模型)融入到传统零售管理系统的架构中,本项目开创性地将针对零售管理相关场景开发的大语言模型推理托管、模型推理接口抽象化(abstraction)功能添加到了服务器中。以计算负载从算力低的节点转移向负载能力高的节点转移(offloading)为核心思想,以统一系统中不同组件利用大语言模型的方式为目标,设计了 \verb|manyshot| 推理库及服务器中对该模块的封装。同样地,为了强化实时推理需求较大的图像分类模型,同时降低部署和使用成本,本设计开发了将图像模型训练、推理、量化(quantization)集合为一体,同时加入可替换优化器和后端的模块化部署功能的 \verb|yolod| 图像分类模型托管服务器。

\subsection{提示词模板的设计和优化}

为了将原本没有特殊调优、并非十分适用于零售商业相关任务的大语言模型(如“千问”、“QwQ”系列模型和“DeepSeek-R1”等)对相关问题的回复与该系统整体设计模板相对其(align),本设计创新性地针对这些通用大模型设计了相应的优化的提示词还有可以根据用户输入进行实际调整、替换和微调的“动态提示词模板”,通过对用户、程序的输入作出各种特定的修正,在没有经过调优(fine-tune)的大模型上达到更加令人满意的效果。

\subsection{中文关键词搜索与无关键词搜索}

为了在潜在的性能和存储空间受到限制的设备上实现部署效果优良、反应迅速的搜索功能,本设计以位置无关的文本关键词匹配为设计思想,通过创新性地引入中文文本分词算法和大语言模型近义词搜寻算法,在仅利用已有的商品信息的情况下实现了复杂度与商品数量、商品文案长度基本不相关的高效搜索算法。

为了在不需要消费者提供关键词或其近义词的情况下实现商品的检索,本设计提出了利用与经系统提示构造的AI“导购”进行多轮对话,再对其回复进行提示词工程信息(关键词)提取来实现的无须顾客提供任何直接的关键词就可以展开搜索的“间接搜索”搜索方法。